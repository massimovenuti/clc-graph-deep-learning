\chapter{Travaux futurs}

La première partie de ce projet s'est principalement articulé autour de l'appréhension et la préparation des données
pour construire les graphes.
La seconde partie consistera en la mise en oeuvre de l'apprentissage automatique pour la classification des graphes à
partir de ce travail.

La prochaine étape sera alors de s'intéresser aux méthodes de plongement de graphes basées sur l'utilisation de réseaux
de neurones.
Pour ce faire, un état de l'art sera nécessaire pour découvrir, comprendre et choisir les méthodes les plus adaptées à
notre problématique.
Il s'agira ensuite de les implémenter, les évaluer et les comparer.

On s'intéressera en priorité aux données des communes de France, l'objectif étant d'appliquer les méthodes retenues à
des échelles différentes, pour classifier des zones de plus en plus grandes et recouvrir finalement l'entièreté du
territoire français.
Cela conduira donc à la création de nouveaux jeux de données, dont la taille des zones géographiques, et donc des
graphes, sera très variable.


