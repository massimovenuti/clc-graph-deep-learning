\chapter*{Introduction}
\addcontentsline{toc}{chapter}{Introduction}

Dans un contexte où les données à analyser sont de plus en plus nombreuses et connectées, la représentation des données
sous la forme de graphes pourrait faciliter l'extraction de liens et de structures communes.
% reformuler ?
Les réseaux sociaux, la cybersécurité et la bio-informatique font partie des domaines où les données se
modélisent naturellement sous la forme de graphes.
%Cependant, l'extraction d'information à partir des graphes nécessite un moyen de les analyser efficacement.
Cependant, l'apprentissage d'une métrique de similarité entre les graphes est considéré comme un problème clé dans le
cadre de tâches telles que la classification, le regroupement et la recherche de similitude.
% à compléter si nécessaire

Récemment, il y a eu un intérêt croissant pour l'utilisation de réseaux de neurones pour estimer la similarité entre
graphes.
L'objectif à travers leur utilisation est de plonger les graphes dans une espace cible, de telle sorte que la distance
dans l'espace cible se rapproche de la distance structurelle dans l'espace d'entrée.
Ce travail d'étude et de recherche vise alors à explorer et évaluer ces méthodes sur les données géospatiales de la base
de données \emph{Corine Land Cover}.

%L'objectif de ce travail d'étude et de recherche est d'explorer et évaluer des méthodes pour estimer la similarité entre
%graphes basées sur l'utilisation de réseaux de neurones

